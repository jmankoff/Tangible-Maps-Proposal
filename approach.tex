Our previous work in accessible mobility and transit as well as in building information technology to communicate and close the informational gaps experienced by travelers with special needs led us to develop a holistic approach to measuring “accessibility of the full travel chain,” involving coordination between transportation departments and peers in human service departments, and employing participatory design principles (in which an active group of participants from the population of interest, in our case individuals who are Deaf-blind and their supporters, actively participate in the design process). We have employed a specific user-empowerment design methodology, and a nuanced model of ‘accessibility’ which takes into account the fact that ‘disability’ is an umbrella term for a truly heterogeneous population and that effective mobility and transit solutions for such a variable population requires inherent flexibility and customization. This approach guides the development iterations, validation, analysis steps, and coordination that our team will implement to ensure successful, comprehensive, value-sensitive design iterations.

The UW team is partnering with the City of Seattle Title II ADA Compliance Program to pursue development and pilot testing of a prototype Tactile Map Tile application in Seattle. To date, TCAT has an excellent working relationship with the City of Seattle. Additionally, Seattle's Lighthouse for the Blind makes an ideal local partner because of its ongoing commitment to accessibility and continual innovation for the city's blind and Deaf-blind population. The AccessMap and OpenSidewalks Projects are both examples of TCAT's commitment to accessibility through partnerships. 

To ensure the success of the Tactile Map Tile development project, we developed two core teams of stakeholders: the Technical Team and the Community of Practice Group (stakeholder group). While some staff resources are shared between the two groups, in the three year development period, we intend to pursue continuous iteration between achieving planned technical goals (described in the development activities in Sections \ref{sec:fabrication}-\ref{sec:alerts}) and learning from real-use evaluation (described in the data collection and community activities in Section \ref{sec:stakeholder-input}).
Therefore, the technical development tasks should be sometimes concurrently occurring with validation testing and translation to practice. This will allow us to outline a process that enables  faster prototyping iterations, mutual learning, including collective reflection-in-action, through trial use of the tactile maps for various trip purposes. In our experience, this type of development allows for emergent and opportunity-based changes to the technical design resulting from trial use of new maps by the Community of Practice. 

