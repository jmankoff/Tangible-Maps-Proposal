Nidler grants must comply with public access to publications requirement
++public access to data
+Data management Plan
++must submit one with proposal to meet public access requirement
++No strict answer to how to make data public, but all data must be public
++Does not count against 50 pages and not peer reviewed
++Should appear as an appendix
++What about Qualitative data?
+++No real answer other than "There are ways"
\jen{does not count toward page limit}

\subsection{Data to be collected}
\begin{itemize}
    \item Our data will include xxx software & map related data \jm{also meta data}
    \item Our  data will also include interview and 
survey data describing interactions with maps and analysis of artifacts from the field. Additional metadata will include coding schemes, memos, experiment protocols,  design recommendations and design ideas and scenarios of our proposed maps and map creation tools.
\end{itemize}

\subsection{How data will be organized, stored, and preserved}
\jm{Indication of whether the awardee will suhttps://www.overleaf.com/project/5c21099c86d1f25f54ca6267bmit the scientific data to ICPSR\footnote{applicants may seek technical assistance from the Interuniversity Consortium for Political and Social Research (ICPSR). ICPSR is the preferred data repository for archiving and sharing of scientific data generated under NIDILRR awards.}}

We will make the source code for our tools   available for public download  using an OSI-approved open source license. 

All human subjects will be anonymized and aggregated to maintain confidentiality 
and protect the privacy of subjects and industry partners. 
Key excerpts of anonymized primary data will be routinely shared 
in the context of publications of this research.  
Final coding schemes, design recommendations, and design prototypes will be shared in the context of publications of this work. Human subjects data that includes privileged and confidential information 
and will be kept secure in accordance with approved IRB protocols. \jm{Provide a plan to address the study participants’ consent process to enable the de- identified data to be shared broadly for future research.
} 
In the event that key data has not been published in formal venues within 
five years of the termination of this grant, 
all unpublished manuscripts will be made publicly available online.

\jen{If applicable, explain why data sharing, long-term preservation, and access cannot be justified.}

\jm{Indicate an estimated cost to implement the data management plan. This cost is allowable as part of the award’s direct costs.}