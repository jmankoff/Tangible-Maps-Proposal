The proposed research will result in substantial data collection and software development. The products of these efforts will be released to the public with unrestricted licenses following open practices. 

\subsection{Data to be collected}
The data to be generated under this project include:
\begin{itemize}
    \item An open-source software package and webpage 
    \item IRB consent forms for all studies 
    \item Questionnaire and survey responses from usability studies 
    \item Notes and recordings from observational studies describing interactions with maps
    \item Notes with analysis of field studies. 
    \item Survey data for remote users of website
    \item Additional metadata will include coding schemes, memos, experiment protocols, design recommendations and design ideas.
\end{itemize}

\subsection{How data will be organized, stored, and preserved}

\textbf{Data and Metadata Standards:} All data will be stored in standard formats (ASCII text, JPEG images, MPEG video, WAV audio) as appropriate. No particular metadata standards exist for this data, nor do we anticipate particular metadata needs. 

\textbf{Methods and policies for providing access and enabling sharing:} For long-term storage, all data will be stored on a centralized computing system maintained by UW, in directories with access limited to project personnel. By default, all project personnel shall have read/write/edit access to the data. Internal data storage backup systems will be provided by UW CSE computing services. Disaster Recovery logistics are covered under the policies of UW’s Computing Center. 

The PI's will continue their history of releasing open-source software. All software will be released and available for public download using an OSI-approved open source license. 

As appropriate, the software will also be released through the Taskar Center's  open source repository under www.github.com. News, data, and publications related to the project will be made available on the research page of the PI’s and Co-PIs’ websites as well as the Taskar Center for Accessible Technology website. All released data will be anonymized by removing any identifying information. All software will be released under an open source, BSD license. 

Software releases of individual software components will occur on a 6-month schedule, or more frequently if appropriate. 

\textbf{Policies and provisions for archiving, preservation, re-use, re-distribution:} All data will be preserved for at least three years beyond the award period. Data gathered for this project may be reused in other, related, research projects conducted by the PI, co-PI or their graduate students. All anonymized data will be shared on ICPSR, adhering to the ICPSR repository guidance.  It is possible that other researchers in computing would be interested in our dataset. All such requests will be facilitated through use of the ICPSR repository.

Plans for archiving and preservation of access: 
Long-term data storage will be handled by UW’s existing procedures. All logs will be anonymized before storage. We will store a copy of the software used to record and process the data with the data itself to aid in future access. Research reports and papers on the project will be available through both the authors' websites and the digital archives of the corresponding publishers (such as ACM or IEEE). 


All human subjects will be anonymized and aggregated to maintain confidentiality  and protect the privacy of subjects and industry partners. Key excerpts of anonymized primary data will be routinely shared in the context of publications of this research.  
Final coding schemes, design recommendations, and design prototypes will be shared in the context of publications of this work. Human subjects data that includes privileged and confidential information 
and will be kept secure in accordance with approved IRB protocols. 
In the event that key data has not been published in formal venues within five years of the termination of this grant, 
all unpublished manuscripts will be made publicly available online.
