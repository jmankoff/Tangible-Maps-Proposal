Our development activities will focus on iterative refinement of the
two main threads of the work, namely the development of high quality
optimal tangible maps, and the development of a web interface that
empowers Deaf-blind individuals to requisition their own maps on the
fly.

With respect to maps (Product 1), we next discuss the long history of tangible map use
for accessibility, and the importance of bringing down the price and
increasing the customizability of maps (Section~\ref{sec:background}. We then discuss our
optimization approach in Section~\ref{sec:optimize}, and the iterative
design approach we plan to take to improve our technology in
Section~\ref{sec:mapping-validation}.

With respect to the web interface (Product 2), we highlight the lack
of end-user control in our review of related work
(Section~\ref{sec:background}). In
Section~\ref{sec:accessmap-extension} we discuss the AccessMap
interface and our plans to extend it. Finally, in Section~\ref{sec:stakeholder-input}, we discuss how input
from key stakeholders (namely our intended users) will be collected to
ensure that the interface proposed in
Section~\ref{sec:accessmap-extension} is correctly meeting their
needs.

We conclude with a discussion of the stage of development
(Section~\ref{sec:stage}). We argue that the work represents a
combination of proof of concept and proof of product. 

%Intro: The solution will be integrated into the existing AccessMap infrastructure.
%(Caspi writes)

\jm{Notes from conversation with Harniss:
Emphasize iteration
Blow smoke about the algorithms
}
\jm{Notes from megan on call: +"Development Activities in appropriate enviroment"
++describe the enviroments that you are testing your "something"
++enviorment differs based on circumstance
++describe to the reviewer}

\subsection{State of the Art in Tangible Map Products}
\label{sec:background}
The greatest challenge in mobility and orientation for visually impaired people is to understand the spatial layout of the targeted environment. Maps are an important tool for sighted navigation, but inherently visual. One solution that has a long history in the Blind and Blind-deaf community is tactile maps. Common approaches include embossing, printing ind Braille, and "swell" (microcapsule) paper. These approaches can be used to create a two-layer tactile visualization from an image (\textit{e.g.}, \cite{miele2006talking}). However such machines can be very costly \cite{rice2005design}, are not readily available, and lack the expressive range of a solution that can vary more in height. While vacuum-forming has more height, it depends on a pre-existing form, which limits its customizability. Finally, the design of a tactile map requires expertise to be easily understandable \cite{tatham1991design}.

With the advent of consumer-grade fabrication technologies, 3D printed and other custom fabricated maps have been receiving increasing attention. For example, it is now possible to requisition a custom, laser-cut topographical map on ETSY \cite{etsy} or purchase custom 3D printed topographic models through companies such as Sightline \url{sightlinemaps.com} or PrintMyRoute \url{www.printmyroute.xyz}. However, these services do not include features intended to support Blind or Deaf-blind users, and do not support activities such as route finding and landmark identification as a result. 

The accessibility community has added its own spin to these technologies. For example TouchMapper \url{touch-mapper.org} is a free service that will create a tactile map using embossing or 3D printing and help connect you to existing online printing services or print it yourself.  This service focues on showing streets, and can be customized for size, inclusion of buildings and other features. However it too does not include route finding or landmark identification. 

From a research perspective, one open problem is combining tactile maps and smartphones. By embedding capacitive touch sensing capabilities in tactile maps, it is possible to provide audio feedback about the region someone is touching \cite{taylor2016customizable, rusu2010semantic,gotzelmann2016lucentmaps}. 

Another open question is which cartographic features should be highlighted in tactile maps \cite{haberling2008proposed}. While tactile interaction has been studied extensively (\textit{e.g.}, see this review \cite{o2015designing}), the specific design considerations for tactile maps are not as well understood, particularly for Deaf-blind individuals. In addition, there is no computational model that takes these variables into account exists.

Finally, questions remain about the ability to of tactile maps to support route finding (as opposed to orientation). For example, Gual \textit{et al.} found that standard 3D printed maps can improve memorization in route finding, but could not be used autonomously without collaborator support \cite{gual2012visual}. However, they did not explore a wide range of tactile variables to support interpretation. 

To summarize, tactile maps are not new to the cartographic record \cite{xx}. Their value in facilitating orientation and navigation for the low vision and blind communities has been well established \cite{XXX}. However, their scope and availability has been greatly limited in the past by high production costs and limited interest from fields traditionally invested in map making and design.  3D printing has significantly reduced the costs of producing such maps, but to our knowledge no existing product has enhanced 3D printed maps with optimization and customization, making our product and important addition to this space. 



\subsection{Product 1: Automatically optimize map design based on specific needs}
\label{sec:optimize}
Route planning requires information that may be specific to the person creating a map. For example, the pedestrian directions from University Street Station on Second Avenue to Seattle City Hall on Fifth Avenue on Google Maps route people up Seneca Street, which has a steep 10 percent grade that’s problematic for people in wheelchairs or with certain injuries or health conditions. %By contrast, AccessMap sends people two blocks north to Pike Street, which has a much gentler grade of less than 2 percent.
%For such people, it is important to make sure that a map clearly indicates slope. For someone else concerned about safety, however, it might be important to show whether there is a guard rail present near a particularly busy street, whether a sidewalk is wide or narrow near a busy street, or whether there is an alternative pedestrian route that connects their sidewalk to a pedestrian footpath removed from the road. 
%For accessibility, 
Users have a variety of needs, so while a powered wheelchair user may need to know where curb ramps are in order to navigate an intersection, a blind user may want to actively avoid certain types of curb ramps (such as those pointing to the center of an intersection). These different informational requirements present a particularly difficult situation to resolve without a specific description of sidewalk and curb ramps locations. This is why optimization is important.
\jm{does this go in introduction instead?. also more about what optimization is/does}

\subsubsection{GAP Algorithm}
\jm{this is way too long and technical and needs to be cut way down}
Maps uses many renditions to represent various forms of information. A single rendition, $r$, could be a type of icon, or a colored line. Each rendition can only represent one class of information in a map. For example, a circular icon could note an accessible intersection, or a guide-dog friendly coffee shop, but it would be confusing if it represented both. Not all forms of information are compatible with all renditions. You can not represent a road with just a circular icon. The binary value stating if a a piece of information (a.k.a., a datum), $d$, is compatible with an rendition, $r$, is denoted $c_{rd} \in \{0,1\}$. If the datum is then assigned to a compatible rendition, this is denoted with the binary value: $x_{rd}\in \{0,1\}$

Our goal is to assign the most relevant pieces of information to the most useful renditions while maximizing the information we present. We  formulate this as an zero-one linear program interpretation based on the Generalized Assignment Problem
\cite{unknown_GAP}:

The zero-one linear program is formulated as follows:
\begin{equation*}
\textrm{ argMax }
\sum_{i=1}^{|R|}
\sum_{j=1}^{|D|}
\alpha(r_i, d_j) x_{ij}
\end{equation*}

\begin{subnumcases}{
\textrm{ s.t. } 
}
   \forall_{i \leq |R|} \forall_{j \leq |D|} r_{ij} x_{ij} \leq r_{ij} \label{data_rendition_compatability}\\
   \forall_{i\leq|R|} \sum_{j\leq|D|} x_{ij} = 1 \label{rendition_Cap} \\
\forall_{j\leq|D|} \sum_{i\leq|R|} x_{ij} \leq 1  \label{unary_data}\\
\sum_{i\leq|R|} \sum_{j\leq|D|} x_{ij} = \min(|R|, |D|) \label{complete_fill}
\end{subnumcases}

\item[Weighting Information-Rendition Pairings]
When weighting a particular pairing of an rendition with we must consider four features. First, is the pairing viable; can the information be represented with the abstraction. If not, the weighting in the objective function for the pairing, $\alpha_{ij}$, should be set to 0 and should not contribute any value to the maximized sum. However, if a pairing is viable we must consider a more complex representation of that pairings value. To do this we model three pairing features with the acronym (CIA): the rendition communicative ability (C), the data's importance (I), and the  attentive cost (A). We combine these by subtracting the attentive cost ($A$)  from the benefits ($C*I$): 
\begin{equation}
\label{eq::CIA}
\alpha(r_i, d_j) = C(r_i, d_j)*I(d_j)-A(r_i, d_j)
\end{equation}

\begin{description}
\item[Communicative Ability]
Communicative Ability, $C(r,d)$ is the measure of how well a rendition, $r$, communicates a specific type of information, $d$. A high level measure of this is already accounted for in the hard constraints of the linear program: if a rendition is incompatible (\ie incapable of communicating a class of information) constraint \ref{data_rendition_compatability} will not be met and the pairing will not be accepted. But our model must support more nuanced expressions of information communication. For instance, there may be many types of renditions that label a coordinate on a map, but each are subject to certain limitations. For example, the number of parallel roads that can be represented in a space is dependent on the width of the lines representing the roads; denser road networks require thinner lines, will sparse networks could make use of more distinguishable, thicker lines. Further, certain attributes of a rendition may be more or less desirable to a specific user, such as the preference for braille or embossed text. 

In our system, all renditions have a set of attributes, $a \in A_r$, (\eg uses braille, uses raised edges vs uses indented edges). Similarly, users profiles, $U$, and classes of information, $d$, have a set of preferred attributes, $\hat{A}_U,\hat{A}_d$ and weights on those preferred attributes $\beta_a$. The communicative ability of an information-rendition pairing is the weighted sum of the intersection of a renditions attributes, an information classes preferred attributes, and a user's preferred attributes. 
\begin{equation}
\label{eq::communicability}
C(r,d,U) = \sum_{a\in A_r \cap \hat{A}_d}(\beta_a) +  \sum _{a\in A_r \cap \hat{A}_U}(\beta_a)
\end{equation}
Attributes can be expressed in many ways, and determining the intersection of a renditions attributes and preferred attributes is managed through a series of adapter interfaces. For instance, one interface is a Brailleable rendition, and which requires the rendition to generate related text in braille. A user profile and information maintains a list of relevant adapters (\ie the preferred attributes) mapped the the preferential weight and any parameters of that preference. An example parameter is a preference for lines no thicker than 2 mm or no shorter than 1mm. These parameters are applied to the rendition through the adapter. 

\item[Information Importance]
Information importance is the simplest measure of an Information-Rendition Pairing and it is central to the adaptive design paradigm our system represents. Users have a good understanding of what information is important to them. They know what accessibility features and challenges effect them most, and what points of interest are most relevant to them. For this measure, we simply ask users to rank information, the higher the ranking, $I(d_j)$, of a class of information, the higher the weight. If the information is useless or irrelevant to the user the ranking is set to zero which intern makes the weight, $\alpha$, on all rendition pairings with this information zero or less, guaranteeing that the information will not be presented. For efficiency reasons, information classes that are marked as having no importance are excluded from linear program a-priori. 

\item[Attention Cost]
The attentive cost of an rendition measures how distracting it is to gather information from the rendition. For example, if a user is using a rendition of a road network to navigate the map in search of a target, say their favorite coffee shop;  the longer it takes to find the target the more information they have to keep track of (\eg where they started, what turns they made, what landmarks they noticed). Tracking all of this information carries an attention cost that would otherwise be spent on the primary search task.

The problem is that when the map is being constructed and the attention cost weight is need we do not know what the user's target(s) will be, where they will start their search, or what other information is presented that may help or hinder the search task. The presentation of that information is, in fact, our goal. Given this level of uncertainty we use a Monte-Carlo simulation to estimate the average \textit{time} it takes for the user to perform a search given a rendition-information pairing. The probability distribution used to generate this Monte-Carlo simulation are derived from a Markov-chain state model representing the actions a user can take when they encounter portions of a rendition. 
\begin{figure}
\label{fig::exampleStates}
	\begin{tikzpicture}
        % Add the states
        \node[state] at (0, 0) (o) {Off Road};
        \node[state] at (4, 0) (r) {On Road};
        \node[state] at (2, 4) (i) {Intersection};

        % Connect the states with arrows
        \draw[every loop]
        	(i) edge[loop left] node {} (i)
        	(i) edge[bend right=20] node {} (r)
            (r) edge[bend right=20] node {} (i)
            (i) edge[bend right=20] node {} (o)
            (o) edge[bend right=20] node {} (i)
            (o) edge[loop left] node {} (o)
            (r) edge[bend right=20] node {} (o)
            (o) edge[bend right=20] node {} (r)
            (r) edge[loop right=20] node {} (r);
    \end{tikzpicture}
\caption{Example of states of navigating a road network}
\end{figure}

Suppose that a user is navigating a rendition of a road network. There are three basic states of the action: (1) moving the finger along a road, (2) encountering an intersection of roads, and (3) moving the finger of the road. Which particular state the user is in is dependent on the specific roads paired to the rendition, and their likelihood of moving from one state to another is dependent on both the particular road network (the information) and the rendition. For example, when the user encounters a an intersection they are very likely to continue onto a new road (\eg changing from the Intersection to On-Road state), but which road is dependent on many factors. Generally they are more likely to move in the same direction and increasingly less likely to turn all the way around. If the rendition presents very thin roads, they are more likely to travel in a direction that they don't feel the roads any more, moving into the "Off-Road" state. We model the probability of this state change by selecting a random angle between $-\pi$ and $\pi$ from a normal distribution with a mean direction of 0 (no change in direction from the approaching road). If the user travels in that direction but could still feel a road (based on the width of the finger and thickness of the rendition), then the angle will be modified to continue along that road, otherwise they will move randomly in that general direction off-road until they find a new road to follow.  

Each state change in the Markov-chain takes about step of one finger width, and we use a state change as a unit of time. For each simulation in the Monte-Carlo model, the user takes a "walk" across the map starting at a random point and moving towards a target. The starting points and targets are selected from a probability distribution where areas denser with information are more likely to be selected. The Markov-chain determines the walk that they take. We count the number of steps in the walk it takes to find the target given the rendition and information. The average number of steps over many models is our attention cost, meaning the more difficult (the longer it takes) it is to find information using a rendition-information pairing, the greater the attention cost and the poorer the pairing. 

In terms of implementation, each rendition, $R$, has a set of states $S_R$. The probability of entering a state, $s_i\in S_R$, is dependent on the data, $D$, paired to $R$ and the state it is entering from, $s_{i-1}$: $P(s_i | D, s_{i-1})$, A walk over the map, $W$, starts from a starting point/state $s_1$ and we randomly change the state based on the probability distribution of all possible next-states. With each state we move the finger, $f$, a one finger width, $w_f$ in the direction dictated by the current state. The attention cost of a walk, $A(W)$, is the number of states need to get the point $f$ within $w_f$ of the target point $t$.The attention cost of a pairing, $A(R,D)$ is equal to the average attention cost of all of $N$ simulated walks.

\begin{equation}
W \subset S_R | s_1...s_{|W|}
\end{equation}
\begin{equation}
A(W) = |W|
\end{equation}
\begin{equation}
A(R,D) = \frac{\sum_{i=1...N} A(W_i)}{N}
\end{equation}
\end{description}

\subsubsection{AccessMaps Integration}
\label{sec:accessmaps-integration}



\subsection{Validation}
Describe tests trials and other related activities, and justify the choice of sample and environment. 
\subsubsection{Study 1: Map Understanding With and Without Optimization}
\label{sec:lab-tests}
Our first study will focus on demonstrating the benefits of optimization in a controlled fashion. \jm{fill in with a description of how we will do a controlled, comparative study of the value of two different maps...}
\subsubsection{Technical Validation}
\sec{sec:technical-validation}

\subsubsection{Study 2: Navigation Benefits of Optimization}
\label{sec:field-map}
\jm{I think we also want a study of them choosing a route of their choice and using our system?}

\subsubsection{Study 3: End-to-end system use}
\label{sec:field-web}
\jm{test that they can produce a map on their own when they want and use it}

\subsection{Input from Stakeholders}
Describe how input will be collected from key stakeholders (including people with disabilities) to guide development activities.

\subsection{Stage of Development and Specific Plan}

This project includes both the \textit{proof of concept} stage and the \textit{proof of product} stage:
\begin{description}
\item[Implementing and Testing Optimization Algorithm (Proof of Concept)] While we have preliminary results demonstrating the viability of optimization for this problems space \ref{sec:optimization}, our proposed work includes recruiting and testing with Deaf-blind individuals \ref{sec:lab-tests}, and resolving remaining challenges in optimization \ref{sec:optimization}.
\item[Testing in Natural Contexts (Proof of product)] Our proposed work also includes field studies. We will be testing the map in use in field settings \ref{sec:field-map}, and testing the entire system (including web-based map creation) in the field \ref{sec:field-web}. 
\item[Fully Integrated Prototype (Proof of product)]
Our final implementation will be integrated into AccessMaps, a publicly available mapping application. \ref{sec:accessmaps-integration}.
\item[Verification of Technical Requirements (Proof of product)]
We will validate the generalizability of our approach \jm{keep this? What is the appropriate technical validation?} \ref{sec:technical-validation}
\end{description}
    


\subsection{Product 1: Validation: Iterative Design of Mapping Solution}
\label{sec:mapping-validation}
Describe tests trials and other related activities, and justify the choice of sample and environment. 

We will take a user-centered, iterative approach to the design of our mapping solution. All of our studies will be conducted with the population we are serving, Deaf-blind users. PI Caspi has extensive experience working with this population \jm{true?} and will be able to help ensure that we have access to participants. 

\subsubsection{Map Study 1: Map Understanding With and Without Optimization}
\label{sec:lab-tests}
Our first study will focus on demonstrating the benefits of optimization in a controlled fashion. For this study, we will use pre-defined maps focusing on neighborhoods familiar to participants. we will ask participants to indicate the location of a well known landmark, and to describe the region around the landmark. We will also ask participants to describe features of a route between two landmarks. We will then have them do the same tasks with an optimized map. \jm{how do we ensure famimliarity? do we care about route finding? How do we make sure the optimization is relevant to them? How do we train them on all the renditions' meanings?+}

\subsubsection{Technical Validation}
\sec{sec:technical-validation}

\subsubsection{Map Study 2: Navigation Benefits of Optimization}
\label{sec:field-map}
\jm{I think we also want a study of them choosing a route of their choice and using our system?}



\subsection{Product 2: Extending the AccessMap interface to 3D printed
map generation (Caspi-accessmap.tex)}
\label{sec:accessmap-extension}
%jen: repeats words in the bcakground section Tactile maps are not new to the cartographic record. Their value in facilitating orientation and navigation for the low vision and blind communities has been well established. However, their scope and availability has been greatly limited in the past by high production costs and limited interest from fields traditionally invested in map making and design. 

% same These maps have been designed as tools that enhance spatial understanding for people within a large range of visual capacities.  They abstract nonvisual cues from the pedestrian environment and consider circumstances that influence a full spectrum of experience. 

\subsubsection{Background and State of the Art}

\begin{comment}
I believe this is used somewhere else. Need to pull in.

Independent navigation is essential for autonomy and community participation in urban centers. 
Navigation solutions for both sighted and blind individuals fall into two important categories -- turn by turn directions, and maps. 
A common solution for Blind navigators is GPS programs that provide turn by turn directions. We tested the three blindness-aware GPS navigation mobile apps recommended by the American Foundation for the Blind: "Nearby Explorer", "The Seeing Eye GPS App", and "BlindSquare", and none offered a simultaneously sparse and informative solutions when used in conjunction with a portable Braille reader \cite{AFBBlindnessNavApps}, often because it is difficult to consume a lot of information portably with Braille readers on the go, and not all the information was equally relevant. 

\end{comment}


\subsubsection{Previous Development}

The Taskar Center has two other relevant projects aiming to improve access to mobility and transportation for individuals with disabilities. The Taskar Center's overarching goal is 
to develop seamless regular-commute transportation customized assistance system that integrates multiple sources of current and critically relevant travel information.

This project build on AccessMaps, which itself depends on data from OpenSidewalks. As we will show, \jm{key points for extending accessmaps}




AccessMaps is \jm{Anat fill in introduction}.
Figure~\ref{fig:accessmap} shows the current version of the system in use. 

\begin{figure}
    \centering
    \includegraphics[width=5in]{pics/accessmap.png}
    \caption{AccessMap Screen Shot}
    \label{fig:accessmap}
\end{figure}

AccessMap currently can produce parametrically designed maps that always show the most up to data open data (derived from OpenSidewalks) about a region. 

\subsubsection{Proposed Development}

Operating hand-in-hand with AccessMap and OpenSidewalks (two projects discussed below), the goal of our work is to extend AccessMap to support users to automatically generate a custom 3D map model of any given area.  That model could then be printed at home, brought to a local library, or sent to a 3D printing service for relatively quick and inexpensive map production.  Beyond customizable map locations, ultimately the application would allow users to specify different scales and map features that are important to them.  

We will base our tactile design features on the comprehensive set of tactile map graphic symbols adopted by Braille Authority of North America (BANA) as created by (\cite{lobben2012tactile}, p. 107).% integrated into data driven design development tools and made available and consumable to landscape architects

%Anat: I commented this text out because I don't think it is product focused enough for NIDILLR's FIP in development. 
%The Tactile Maptile project designed the set of associated atomic symbols for that critical information.  Not only does this have implications for the map tiles specific to this project, but also more broadly works towards elevating pedestrian infrastructures in the context of our digital landscape.  This is important from a navigational perspective, but is also a reflection of the information designers, planners, and policy makers depend on to make significant decisions that affect our urban fabric.  Informed decisions based on incomplete data are not only difficult but also more prone to error and bias.  As we move rapidly towards sensor laced smart cities, it is critical these gaps be identified and understood in order to account for this influence on design and decision-making.  As such, this work is meant to take an accessible approach to data as it relates to pedestrian design and experience.

%Illustrative documentation of both the process and analysis that went into making these maps is directed more squarely at the design community. This project re-examines the pedestrian environment, with a focus on the specific needs of the low vision and blind communities. The goal of this work is to persuade designers to consider a broader spectrum of experience, and engage more critically in what it means to be designing inclusive cities.  

%This project is meant to bring attention to the deficiencies of the system currently place, in which accessibility checklists too often are accepted in lieu of truly inclusive design.  The straightforward approach is intended to remind designers that accessible design is good design, and if we want to build more equitable cities that means there is a huge spectrum of experiences we must first understand. 

\subsubsection{Validation}

\ac{ needs writing


Our solution combines simple accessible interfaces with complex data integration and smart routing. To our users, the entire solution will be seamlessly presented in a workflow through which the travelers access a website, select the area of travel, the type of travel they wish to undertake in the area and travel preferences. The travelers are given the opportunity to verify the map location and features \textit{via} non-visual text-based exchange before printing the map.
% means what? A: means that in our UI pilot we found one of hardest things with building this UX/UI is ensuring that the tactile map model they got isn't of [Paris, Texas] when they actually meant [Paris, France]

The entire exchange is enabled and specifically designed for a portable 14-cell Braille-display. At the end, the traveler receives access to a downloadable 3D model file, access to an online URL where the model can be accessed for a specified duration, the option to have the model printed and sent to the user for a fee, and the option to subscribe to email alerts regarding any changes to the mapped region in the digital map repositories. 

}

\subsection{Input from Stakeholders}
\label{sec:stakeholder-input}
Describe how input will be collected from key stakeholders (including people with disabilities) to guide development activities.

\subsubsection{Study 1: Focus Group around mapping needs}

\subsubsection{Study 2: Usability of Interface}

\subsubsection{Study 3: End-to-end system use}
\label{sec:field-web}
\jm{test that they can produce a map on their own when they want and use it}


\subsection{Stage of Development and Specific Plan}
\labe{sec:stage}

This project includes both the \textit{proof of concept} stage and the \textit{proof of product} stage:
\begin{description}
\item[Implementing and Testing Optimization Algorithm (Proof of Concept)] While we have preliminary results demonstrating the viability of optimization for this problems space \ref{sec:optimization}, our proposed work includes recruiting and testing with Deaf-blind individuals \ref{sec:lab-tests}, and resolving remaining challenges in optimization \ref{sec:optimization}.
\item[Testing in Natural Contexts (Proof of product)] Our proposed work also includes field studies. We will be testing the map in use in field settings \ref{sec:field-map}, and testing the entire system (including web-based map creation) in the field \ref{sec:field-web}. 
\item[Fully Integrated Prototype (Proof of product)]
Our final implementation will be integrated into AccessMaps, a publicly available mapping application. \ref{sec:accessmaps-integration}.
\item[Verification of Technical Requirements (Proof of product)]
We will validate the generalizability of our approach \jm{keep this? What is the appropriate technical validation?} \ref{sec:technical-validation}
\end{description}
    

%\subsection{Enabling User to Produce Maps Themselves (Mankoff)}

\subsection{Conclusion}
needed?
% LocalWords:  customizability
