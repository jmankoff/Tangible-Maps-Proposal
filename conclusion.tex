
Mobility and orientation are among the biggest challenges for visually impaired people.
This project addresses a critical need of making available information about mobility and transportation to a cohort that has been excluded from a growing body of timely knowledge on which the majority of citizenry rely to access mobility.
This project presents an alternative approach to understanding the pedestrian experience of individuals who have both vision limitations and impacted hearing, a growing subpopulation among the aging population. 
We have learned from our prior work that it is important to identify scalable ways in which pedestrian environments and transit routes are encoded in tactile forms, and given the specific constraints of the population, it is important to provide a simple, cost-effective way to accessing this information.
Through optimizing the tactile readability of the maps, we can help change the pedestrian navigation experience and assist in planning for The Complete Trip. Our system is a tool that will enable users to create their own maps independently, and to navigate independently. This in turn will maximize their independence and improve their inclusion into society, access to employment, and ultimately their economic and social self-sufficiency. 
