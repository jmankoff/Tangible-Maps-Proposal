to connect map production to existing online printing services. 
(Caspi /Mankoff writes; this is based on the tactile maptiles texxt on her website)

We will develop  a web application that will allow users to automatically generate and customize a 3D map model of any given area. The end product will be a customized map tile, that users co-design by making personal choices about the location, scope, and selection of environmental features.

The underlying data for our end-to-end system will be based upon AccessMaps, a project/data set provided by our partners at the Taskar Center (TCAT) which enables accessible, safe sidewalk trip planning for people with limited mobility. Transportation routing services primarily designed for people in cars don’t give pedestrians, parents pushing bulky strollers or people in wheelchairs much information about how to easily navigate a neighborhood using sidewalks. AccessMaps addresses this by assisting people with disabilities in planning their routes. The application incorporates mapping, GIS data, municipality-specific data, transportation information, and eventually weather and other state-specific information like construction. AccessMaps provides an online travel planner offering customizable suggestions for people who need accessible or pedestrian-friendly routes when getting from point A to B in Seattle.  AccessMaps currently includes support for four cities: Philly, DC, NY, and Seattle \jm{Anat:: is this correct?}

Our work will extend AccessMaps with an accessible interface for generating custom 3D printed maps. 

%OpenSidewalks improves mapping of the pedestrian network within Seattle by importing open data on sidewalks and curb ramps from the City of Seattle's open data platform, data.seattle.gov. This original source data has extensive coverage (the entire city of Seattle), but consists of inaccurately-generated offset lines, so the import   uses processed city data donated by accessmap.io. This data is available under the public domain and has no licensing restrictions.

%OpenSidewalks is based on OpenStreetMap, which is chosen for the back end of our service, because of it’s extensive global coverage and easily extendable data schema. This platform allows for the project to easily pull from a large existing data pool, while also making it simple enough to fill in informational gaps as they pertain to pedestrians.


