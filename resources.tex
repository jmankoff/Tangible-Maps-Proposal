\jen{Are the available resources (labs, etc) sufficient to do the work. Are the facilities accessible? What about online resources?}

\subsection{Taskar Center}

The Taskar Center for Accessible Technology (TCAT), housed by the Paul G. Allen School of Computer Science at University of Washington, develops, translates and deploys open source universally accessible technologies, with a focus on benefiting populations with motor limitations or speech impairment.
TCAT's academic mission is to engage undergraduate design and engineering students in participatory design and inclusive design practices.

TCAT focuses on
pedestrian access and access to transportation;
universal access to collaborative work/play environments; and 
computational tools to support community engagement in the creation of access solutions.

\subsection{Fabrication facilities}
Fabrication resources are centered in three laboratories – the \textit{Ubicomp} lab, which has a high-end electronics space, and the soon to be completed \textit{Make Lab }(ETA January 2019). In combination, these spaces will include a carbon fiber printer, multiple high end powder bed printers, laser cutters, a Shima Seiki machine knitting machine, multiple consumer-grade fused deposition printers, a wood workshop, a large and small object assembly space and other similar resources.   

\subsection{User Interface Studies Resources} 
Research in user interfaces, centered in the HCI Laboratories, utilizes a set of high-end graphics workstations, two multiple-node compute clusters with GPU processing, and a variety of special-purpose devices. Further, there is support for data mining thanks to an an intensive compute server, a Beowulf cluster composed of 37 dual processor nodes with gigabit Ethernet interconnect. All processors are high-end Pentium 4 Xenon processors, the total memory of the nodes is 216GB, and the cluster has a total of 9.6TB of disk space. A second cluster consisting of 21 dual-processor nodes, with 84gb of memory and 5.3 TB of disk space supports AI research. In addition to computing resources, there is dedicated space for conducting user studies and analyzing data. 

\subsection{General Resources}
The Allen School maintains a wide variety of state-of-the-art computing facilities for research and instructional use. The Computer Science Laboratory coordinates the acquisition, maintenance, and operation of the computing equipment and network services. General-purpose research computing is provided by over 900 Windows and Unix-based workstations and servers, located in laboratories, machine rooms and offices. The back-end infrastructure is comprised of general-purpose compute, file, web, mail and print servers, operating as a well-integrated Linux and Windows 7 environment. In addition, around a dozen compute clusters are used by a range of research projects. School networking utilizes 1 and 10 gigabit Ethernet connections to servers and desktop machines, and a dual-band wireless network provides 802.11b/g/n connectivity throughout the building and in surrounding exterior areas. Several large plasma screens and a 56" HDTV provide high-definition video display for networking and graphics research and for video conferencing.

\subsection{Resource Accessibility}
The \textit{Make Lab} is designed with accessibility in mind, including plenty of space for maneuvering, moveable carts for holding projects, and variable height desks as well as a lab manager who can facilitate access needs. The user study laboratories are also fully accessible.

