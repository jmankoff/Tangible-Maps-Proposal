We will take a user-centered, iterative approach to the design of our mapping solution. All of our studies will be conducted with the population we are serving, Deaf-blind users. PI Caspi has extensive experience working with this population \jm{Anat: true?} and will  help to ensure that we have access to a sufficient number of participants. 

\subsubsection{Map Study 1: Map Understanding With and Without Optimization}
\label{sec:lab-tests}
Our first study will focus on demonstrating the benefits of optimization in a controlled fashion. For this study, we will use pre-defined maps that focus on neighborhoods familiar to participants. We will ask participants to indicate the location of a well known landmark, and to describe the region around the landmark. 

\paragraph{Sample}
Our primary goal in this study is to understand whether someone without vision can correctly use the tactile information in the maps. Although this project aims to improve the transit experience for Deaf-blind participants, the specific issue of map interpretability can be tested just as effectively with people who are Blind, which is our intention. In this way, we will reduce the burden on the smaller local population of Deaf-blind individuals by focusing on them in later studies, where their unique perspectives are more critical.

\paragraph{Environment}
The study will be conducted in a lab environment. Because we are focusing on readability, this study will be more effective if we can control for events that might arise in the field, such as distractions.

\paragraph{Test trials}
We will ask participants to describe features of a route between two landmarks. We will then have them perform the same tasks using an optimized map. \jm{how do we ensure familiarity? do we care about route finding? How do we make sure the optimization is relevant to them? How do we train them on all the renditions' meanings?+}

\subsubsection{Map Study 2: Technical Validation}
\label{sec:technical-validation}
Our second study will focus on validating the algorithm's accuracy and complexity. 

Specifically, we will xxx

\jm{is this necessary? Could do speed tests and/or reliability, i.e. does it vary wildly or more smoothly. Megan?}

\subsubsection{Map Study 3: Navigation Benefits of Optimization}
\label{sec:field-map}
\jm{I think we also want a study of them choosing a route of their choice and using our system?}
The goal of our final study is to demonstrate the usefulness of our maps in the field. For this study, we want to increase realism so we will be more likely to discover unexpected variables that may impact the success of the product. 

\paragraph{Sample}
We will work with Deaf-blind individuals to test the maps. 

\paragraph{Environment}
The study will take place in an outdoor environment where all of the complexities of real world navigation come into play. In addition, we will study map use for each participant in a location of their choosing.

\paragraph{Test Trials}
For the first trial, participants will be asked to identify two known locations within walking distance of each other. We will create a map that includes both, optimized based on participant preferences. Participants will navigate from one location to the other, using the map. This represents a relatively easy task, since it is a known navigation task, but will give participants a chance to tell us something about how the map enhances their experience.

For the second trial, participants will be asked to explore the region around a new location. \jm{does this make sense? What is the normal process by which participants orient themselves?}. xx more about this task.