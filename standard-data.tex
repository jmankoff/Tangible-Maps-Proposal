
This Appendix includes an overview of features the OpenSidewalks data project collects pertinent to blind, low vision and Deaf-blind pedestrians. Surfaces are explored first, followed by topography, which is nearly as ubiquitous.  Street interfaces, while occupying less space in a pedestrian environment, arguably have the greatest impact on pedestrian safety and well-being and are the third feature set considered.
This section concludes with a brief overview of additional feature information, namely, transit stops, barriers, landmarks and points of interest; these topics, outside the scope of this work, nonetheless warrant further mapping and design research to enhance independence, mobility and safety for blind, low vision or Deaf-blind pedestrians.

\paragraph{Surfaces}
Surface materials have many distinct traits that are perceptible through the various senses.  Many significant non-visual characteristics affect any pedestrian’s experience, visually impaired or otherwise.
%SK: One could argue that smoothness and evenness (described below) are also visual characteristics. And why do you begin a discussion of surfaces by assuming cane use? Aren't wheelchairs also affected by smoothness and evenness? Should you discuss importance of smoothness and evenness in general before describing a specific use-case?
\begin{description}
    \item[Smoothness.]
Haptic canes make apparent even relatively minor changes in texture (like that found in stamped concrete).  Smoother surfaces require less effort to navigate with a cane.  
    \item[Evenness.]
Like smoothness, evenness can affect ease of navigation with a cane.  In addition, uneven surfaces have a higher potential to be a tripping hazard. 
    \item[Traction.]
Some surfaces are more prone to slipperiness than others. This can be a permanent feature of a material (e.g., a metal plate provides less traction than a gravel path) or a transient property, determined by seasonal (fallen leaves) or weather conditions (sleet).
    \item[Infiltration.] 
Without being able to identify potential water hazards in advance and determine an alternate route, the odds of inadvertently ending up with wet or damaged clothing increase.
    \item[Tactile Paving.] 
High-contrast, truncated dome tactile paving, most commonly found in the United States, is typically used for curb-ramps, stairs, and transit platforms. A variety of other textures -- including cones, bars and decorative paving -- can also be used for wayfinding or to alert pedestrians to a potential hazard.  
    \item[Width.] 
Wide (min 6’-12’) unobstructed paths are one of most straightforward ways to address access in a pedestrian environment (Goltsman and Iacofano 2007, Mitchell and Burton 2006).  They allow sufficient room for navigation with a haptic cane or assistive device and provide a buffer to adjacent traffic.  
%SK:  Not unless a physical buffer also exists, unless I misread the preceding sentence.
    
\end{description}


Ultimately, surface materials should be determined based on a path’s intended use.  To facilitate mobility, relatively smooth, unobstructed surfaces that are easy to navigate and provide adequate traction are best.  Textures can support spaces that are meant to do something more than simply move people through.   Material changes and textures can slow pedestrians down and inform them of transitions, hazards, or other circumstances that warrant attention.

\paragraph{Topography}
Grade change is experienced by everyone in even the flattest of cities on the flattest of blocks.  This work encourages designers to consider how topography affects the pedestrian experience at multiple scales.

\begin{description}
    \item[Macro.]
Despite the best efforts of some engineers (Klingle 2007), steep hills are often unavoidable elements of the urban fabric.  Significant topographical changes can enhance spatial understanding (Gardiner and Perkins 2005), but they nonetheless may pose a barrier to mobility. It is useful to consider ways in which this information might be mediated to the pedestrian's benefit.      
    \item[Transitional.]
Stairs and ramps help pedestrians negotiate site scale macro topography 
%SK:  huh?  What does "site scale macro topography mean"?
and provide common transitional elements to entrances for sites of interest.  Handrails and raised edges facilitate safe navigation.

    \item[Micro.]
Sidewalk cross-slope, curb heights, tree wells, planting beds, and other such features involve minor grade changes in urban environments.  While these may be empirically minor relative to a steep slope or staircase, they are no less significant when it comes to navigating as a pedestrian.  Edge detection, heading and orientation should be considered whenever a grade change is introduced.  If a non-intuitive change cannot be avoided, texture change or other signals can help to minimize conflict and injury.   

\end{description}


Topography often presents a challenge for designers, but accommodating grade change can also be seen as an opportunity to provide information to pedestrians as they move through space.  Edge detection is particularly important when designing with blind or low vision pedestrians in mind.  Potential signals that an edge is near may include the presence of railings, curbs and plantings, or a change in surface material. 

\paragraph{Street Interfaces}
Street interfaces are among the most dangerous spaces in the pedestrian built environment.  Advance knowledge facilitates safety, particularly for people who are blind or have low vision.
\begin{description}
    
\item[Curb Ramps.]
Well-designed curb ramps provide access to pedestrians that rely on assistive devices, but they also facilitate mobility for parents with strollers, cyclists, workers navigating with carts and countless other people with a vast range of circumstances and gear. Newly constructed curb ramps are typically outfitted with tactile pavement that alerts pedestrians to the transition into a vehicle-designated space.

\item[Directionality.]
Curb ramp directionality provides heading information to pedestrians as they exit a protected pedestrian area. As such, a curb ramp must guide the pedestrian into a safe crossing area.  Single curb ramps placed on a corner, or only on one edge of the sidewalk, risk directing pedestrians into traffic.  
%SK:  Why do "single curb ramps placed on a corner" direct pedestrians into traffic?  

\item[Shoring.]
Shoring, or a curb ramp’s raised edge, goes a step further than traditional curb ramps.  It provides an explicit safe heading, which offers navigational assistance to pedestrians using a white cane.  

\item[Crossings.]
For clarity and simplicity, crossing types have been grouped according to the following eight characteristics.  While these characteristics may exist independently, they are not mutually exclusive.  It is also important that these crossing features be considered in context with the street environment (e.g., number of traffic lanes, roundabouts, non-perpendicular intersections, etc.).  

\item[Unmarked Crossings.]
By far the most common crossing type is not marked at all. Any two sidewalk ends or corners separated by a street could be considered an unmarked crossing in certain locales.  Most intersections in residential and industrial neighborhoods fall into this category. Additionally, paint does not last forever: in many places throughout Seattle, marked crossings have all but disappeared as a result of weather and traffic.

\item[Marked Crossings.]
This term encompasses several crossing types that are characterized by a visual cue designed to inform both pedestrians and motorists that a pedestrian may be entering the street area.  Line styles can have different meanings depending on location; however, as a general rule, high contrast markings with greater surface area are easier to detect for both pedestrians and motorists. 

\item[Pedestrian Light Signals.]
This style of crossing signal provides pedestrians with a visual indicator of when it is safe to cross and is typically associated with a corresponding traffic light.  Some of these lights cycle automatically, while others require activation with a button.  In the City of Seattle, the buttons associated with this type of signal are typically rounded and require greater force to push than newer Accessible Pedestrian Signals (APS).

\item[Audible Signals.]
Audible pedestrian signals communicate the state of a crossing interval.  Modern signals are typically associated with a rapid ticking, or a verbal cue such as “walk sign is on” or “wait.”  Older “cuckoo-chirp” signals also fall into this category: a “cuckoo” sound is associated with safe north-south crossing, while “chirps” indicate an east-west pedestrian signal. Cuckoo-chirp style signals are no longer recommended since they tend to confuse pedestrians (Harkey et al. 2010).   

\item[Audio-Tactile (Vibrating) Signals.]
Modern accessible pedestrian signals include a haptic indicator embedded in the crossing signal's push button.  This type of signal is particularly useful for the deaf-blind community, but it also provides a clearer indication of what direction is safe to cross for pedestrians who are blind or have low vision relative to audible overhead signals.  In Seattle, audio-tactile APS systems are typically associated with flat push buttons that require less force to activate than the older systems.

\item[Raised Pedestrian (Tabled) Crossings.]
This increasingly popular type of crosswalk is kept at grade with the sidewalk, encouraging cars to slow down and giving pedestrians priority.  However, in the absence of the grade change that typically signals to pedestrians that they are entering a car-dominant area, it is important to provide a tactile cue through a surface change or warning strips to pedestrians who are blind or have low vision. 

\item[Traffic (Refuge) Islands.]
These painted or raised areas can be located between large traffic lanes, or at the junction of multiple streets intersecting in a non-perpendicular fashion.  Ideally, they  are accompanied by additional pedestrian safety features such as: high contrast tactile paving, shoring edges to provide heading, signage and pedestrian signals.

\item[Pedestrian Bridges and Tunnels.]
Pedestrian bridges and tunnels mitigate traffic conflicts by routing pedestrians on a separate plane or z-level.  To promote accessibility, these features must address the needs of those using assistive devices and make clear both their presence and the locations of their entrances and exits, if separate from the path itself.
\end{description}

A pedestrian’s experience of any street interface depends significantly on the vehicular environment.  Tactile map design should consider ways to facilitate safety and mobility from the pedestrian's perspective, and especially consider the motorist's ability to see a vision-limited traveler. Traffic-calming strategies, such as enlarged corner bulbs, are preferable when optimizing a route for a blind individual. 
%SK:  Does reader know what "corner bulbs" are?


\paragraph{Transit} 
Public transit provides access to the city for millions of riders each year.  Predictably, far side bus stops are preferred in Seattle (Seattle 2005). 
%SK:  What are "far side bus stops"?  I'm thinking a small Glossary would be a big asset for this (and future) proposals.
However, these are not ubiquitous. Being aware of alternative options helps designers better accommodate these inconsistencies in the urban fabric and facilitate understanding for riders. With more information made available, travelers are more likely to attempt unfamiliar trips on public transportation (Campbell et al. 2014).  

Given the importance of transit information, the maps produced 
%SK:  what 'maps produced"? By whom? The "current DOT transit maps present..."?
represent transit stop locations; however, spatial constraints restricted the granularity of data that could be associated with these stops. Ideally, critical information would be available through a linked app that could read bus numbers and location details to the user.  At the very least, this more detailed information should be available at bus stops.  Unfortunately, Braille is notably absent from the vast majority of informational signage at transit stations. 

\paragraph{Barriers, Landmarks and Points of Interest}
Barriers (construction sites, low hanging branches, sandwich boards, pavement cracks…), landmarks (fountains, play fields, churches…) and points of interest (retail, food, services...) are additional categories that should be considered in relation to wayfinding, mobility and the pedestrian experience.  The vast range of potential features, as well as their subjective nature, precluded them from an in-depth investigation in this study.  However, there is precedent and opportunity to explore these features in future work.   

