\jen{Does the applicant encourage applications for employment from persons who are members of groups that have traditionally been underrepresented based on race, color, national origin, gender, age, or disability}

\jen{Are the proposed staff capable of doing the work.}

\subsection{Anat Caspi}

\subsubsection{Capability in Project Domain}
Anat Caspi is Director of the Taskar Center for Accessible Technology at the University of Washington. Dr. Caspi has extensive experience in the development, testing and evaluation of inclusively designed technology. She led the development of the OpenSidewalks data standard, currently the standard of choice for performing paratransit audits by two transit agencies. Furthermore, Caspi has used that standard to develop and deploy the AccessMap automated routing application, which supports non-motorized navigation throughout a number of cities. Her work includes all major aspects of the challenges faced by developing technology for improved access to mobility and transportation. She is currently PI on an effort to develop the standard evaluation framework applied to all projects performed under the Federal Highway Administration's Accessible Transportation Technologies Research Initiative (ATTRI). 

 
\subsubsection{Encouraging Persons from Diverse Groups}

\subsection{Co-PI Mankoff}
P.I. Mankoff is the Richard E. Ladner Professor in the Allen School of Computer Science and Engineering 
at the University of Washington. She applies technologies such as data science and fabrication to 
improving inclusion in and accessibility of our digital future.  One of her primary application domains 
is revolutionizing the production and delivery of 3D printed assistive technology (\textit{e.g.,} \cite{Mankoff:2018:consumer,Hofmann:2016:HelpingHands,Chen:2016:Reprise}).

\subsubsection{Capability in Project Domain}
Jennifer has been a leader in the field of 3D printed assistive technology almost since its inception. Her work includes tool to facilitate the design of 3D printed devices \cite{Hofmann:2016:HelpingHands,Chen:2016:Reprise}; studies of how to integrate 3D printing into volunteer \cite{Parry-Hill:2017:Fabricators5} and clinical \cite{Hofmann:2016:Clinical} settings; and case studies highlighting how 3D printed assistive technologies might be received \cite{Hofmann:2016:HelpingHands}. Among her most relevant work are 
\begin{itemize}
    \item  A novel approach to creating tangible overlays for appliance screens that improve their accessibility \cite{Guo:2017:Facade}. The entire process can be accomplished by a blind person, from photographing the appliance (which is then analyzed by crowd workers) to ordering and attaching the overlay.  to two projects designed to improve the experience received her 
\item A study of the practices that visually impaired individuals use to learn about their environments which outlines requirements for independent spatial learning \cite{Banovic:2013:spatial}
\end{itemize}

Prof. Mankoff has been working in the field of accessibility since before she received her Ph.D.  at Georgia Tech. She teaches a quarterly accessibility seminar and has also taught longer courses on Assistive Technology and User Centered Research and Evaluation. She has also been chair of the SIGCHI Accessiblity community since 2014. She sits on the community council for the e-NABLE community \textit{http://e-nable.org/}, a group of volunteers that work together to 3D print prosthetic hands world wide.

\subsubsection{Encouraging Persons from Diverse Groups}
Mankoff has mentored over 100
undergraduates. She estimates that 59\% are women (an under-represented category in computer science), 6\% latinx/African American, and 4\% disabled. Her group also includes (or has graduated) 14 Ph.D. students. Among these 14, 9 are female, 2 have a disability, and 1 is African American. 

These numbers reflect the fact that Prof. Mankoff regularly  leverages  historical knowledge of and connections to under
represented communities during the recruitment process to ensure that a representative group of students is considered for  positions. Similarly, her mentoring is designed to encourage and support students from varied backgrounds and ensure their success. 
