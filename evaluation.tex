
\label{sec:evaluation}

Our evaluation plan is informed by the work we are concurrently pursuing in defining evaluation frameworks for all ATTRI Development Projects. While the standard is not yet publicly available, we believe the rubric presents a useful guide in pursuing a thorough evaluation of the technology. This evaluation plan will be fully followed twice during the project period, each time after one of the two development phases.


\subsection{Defining Tests}
We begin by defining the tests involved in a full evaluation. 
\begin{description}
\item[Technology Unit Test]
This test is designed to test the technology under different circumstances to (1) Ensure the technology works according to specification; (2) Ensure edge cases are handled properly; (3) Ascertain error and fault tolerance; (4) Test the optimization algorithm
\item[]
\item[]
\item[]
\item[]
\item[]
\item[]
\item[]
\item[]

\end{description}
\begin{description}

\item[Implementing and Testing Optimization Algorithm (Proof of Concept)] While we have preliminary results demonstrating the viability of optimization for this problems space \ref{sec:optimization}, our proposed work includes recruiting and testing with Deaf-blind individuals \ref{sec:lab-tests}, and resolving remaining challenges in optimization \ref{sec:optimization}.
\item[Testing in Natural Contexts (Proof of product)] Our proposed work also includes field studies. We will be testing the map in use in field settings \ref{sec:field-map}, and testing the entire system (including web-based map creation) in the field \ref{sec:field-web}. 
\item[Fully Integrated Prototype (Proof of product)]
Our final implementation will be integrated into AccessMap, a publicly available mapping application. \ref{sec:accessmap-integration}.
\item[Verification of Technical Requirements (Proof of product)]
We will validate the generalizability of our approach \jm{keep this? What is the appropriate technical validation?} \ref{sec:technical-validation}
\end{description}


Assessment should be used to improve the performance of the project through the feedback generated.


\subsection{Assessment of viability}
Cost and time needed will not exceed existing services; 
