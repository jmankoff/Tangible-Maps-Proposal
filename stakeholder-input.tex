\label{sec:stakeholder-input}
In this section we describe the activities of our Community of Practice, our ongoing participatory design stakeholder group. We describe how input will be collected from key stakeholders (including people with disabilities) to guide development activities.

While input from several expert users was incorporated throughout the design process of both the maps and the web application, a formal review of the maps produced thus far has not been conducted. 


The Community of Practice Activities will result in data collection of performance metrics that will allow us to validate the specific development goals as well as perform the overall project evaluation.

The following Community of Practice activities will be pursued within the development period:

\subsubsection{Early Development Baseline Survey}
\label{test:baseline}
Many of the individuals who are Deaf-blind and registered with the Deaf-Blind Service Center in Seattle are already familiar with our tactile map project. To assess our project impact beyond the local region, we will send out a survey to as many individuals as we can reach with the goal of understanding the current reach, cost, accessibility, reproducibility and information content in the tactile maps currently available to Deaf-blind travelers. We will also survey participants about other ways they access geographical, landscape and transit information. 

In this survey instrument, participants will be asked for their:  Age Bracket; HH Income Bracket; Disability Status; HH Size; Number of times tactile maps were used for trip planning in last month; Number of times tactile maps were used for trip planning ever; Cost of tactile maps; Time to creation of tactile maps (if they've had them created for them); Perceived cost of tactile maps; Perceived utility of tactile maps; Perceived eagerness to use tactile maps if they were accessible; Perceived need for improved navigational tools; Perceived need for improved connectivity; For the last 5 trips outside their home: Origin, Destination, Trip Purpose, Departure Time, Number of Modes Used, First-mile Mode, Last-mile Mode;

\subsubsection{Surveys of user groups: alpha and beta population}
\label{test:usersurvey}

These surveys will be implemented twice through the development period, with an alpha populations of 10 users and a beta population of at least 20 users.  
The content of the survey will generally be the same both times. The material covered by the survey is indicated below:
Individual travel patterns
 Age Bracket; HH Income Bracket; Disability Status; HH Size; Number of times tactile maps were used for trip planning in last month; Number of times tactile maps were used for trip planning ever; 
 Basic travel needs including:
Home Location
Up to three common destinations
Response to the user interaction
Response to accessibility, reliability and cost 
Solicited input on how outputs could be improved
Solicited input on how interaction could be improved
Response to the presence of customization options in the Tactile Map Tile planner
Perception of utility of real-time information presented by the update alerts
Perception of utility of information to overcome transportation and pedestrian challenges
Perception of accessibility of information 

Data Collection Period:
The Alpha User Group will be surveyed once the optimization algorithm development is completed and users can produce different maps for exploration only.
The Beta User Group will be surveyed once the production development of the pipeline is completed and the primary development of the optimization is completed for all trip types.

\ac{

In-person interviews
Understanding of information content in the maps
improvements in understanding, mental model of the terrain, satisfaction and feedback. Also used as metrics would be survey questions designed to assess the perception of accessibility and reproducibility
}

\subsubsection{Observational Field Studies: Alpha and Beta populations} 
\ref{test:fieldtest}
These observational field studies will be implemented twice through the development period, with an alpha populations of 10 users and a beta population of at least 20 users. These will be the same populations surveyed by the survey tool in Section \ref{test:usersurvey} so we can match their experience with their qualitative opinion of their experience.

We will invite members of Seattle’s Deaf-blind community to sessions at locations they frequent in the pilot areas. 
Participants will be asked to use the interface to make 4 different maps with different parameters and type of travel designation for each of those maps. 


Participants will then be asked for feedback on several map styles, and will also be surveyed on their informational needs and preferences.  


our proposed work includes recruiting and testing with Deaf-blind individuals \ref{sec:lab-tests}, and resolving remaining challenges in optimization \ref{sec:optimization}.
\item[Testing in Natural Contexts (Proof of product)] Our proposed work also includes field studies. We will be testing the map in use in field settings \ref{sec:field-map}, and testing the entire system (including web-based map creation) in the field \ref{sec:field-web}. 


\subsubsection{Midway Input: Presentation of Working System}

\jm{xx describe some sort of midway input opportunity} 

\subsubsection{Summative Input: Presentation of Final System}

\jm{explain how our final report will include some sort of focus group at the end with reactions to what we've done?}