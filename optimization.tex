
\subsubsection{Background and State of the Art}

Route planning requires information that may be specific to the person
creating a map. For example, Google Maps routes pedestrian directions
from University Street Station on Second Avenue to Seattle City Hall
on Fifth Avenue up Seneca Street, which has a steep 10 percent grade
that is problematic for people in wheelchairs or with certain injuries
or health conditions. 
%SK:  how about older people?  That's not a "health condition" to me :).
%In contrast, AccessMap routes people two blocks north to Pike Street, which has a much gentler grade of less than 2 percent.
%For such people, it is important to ensure that a map clearly indicates slope. For those concerned about safety, however, it might be more important to show whether a guard rail is present near a particularly busy street, whether a sidewalk is wide or narrow near a busy street, or whether an alternative pedestrian route connects their sidewalk to a pedestrian footpath removed from the road. 

%For accessibility, 
%users have varied needs.
A powered wheelchair user may need to know the location of curb ramps to navigate an intersection. A blind user may want to actively avoid certain types of curb ramps (such as those pointing to the center of an intersection). These different informational requirements present a particularly difficult situation to resolve without a detailed description of the available paths, how they are connected, and particular attributes (like curb ramps and their locations) \cite{bolten2017}. Furthermore, it is not possible to represent every possible piece of important information on any map. This is even more pronounced for tactile maps, which can only represent a small amount of meta data beyond the basics of two-dimensional layout using features such as height, texture, and shape. 

%SK.  Start discussion here.
\textit{Optimization} algorithms makes tactile maps accessible and actionable.  Here, we describe both: (1) optimizations that balance information richness with tactile usability, and (2) optimization cost functions that allow customization by Deaf-blind users of varied aspects of their travel experience.
%SK:  Now define optimization and what it does.

\subsubsection{Previous Development: Optimization Approach}
Maps use many renditions to represent various forms of information. A  rendition could be a type of icon, or a colored line. Each rendition can represent only one class of information in a map. For example, a circular icon could note an accessible intersection, or a guide-dog friendly coffee shop, but it would be confusing if it represented both. Not all forms of information are compatible with all renditions. You cannot represent a road with just a circular icon. 
%The binary value stating if a a piece of information (a.k.a., a datum), $d$, is compatible with an rendition, $r$, is denoted $c_{rd} \in \{0,1\}$. If the datum is then assigned to a compatible rendition, this is denoted with the binary value: $x_{rd}\in \{0,1\}$

Our goal is to assign the most relevant pieces of information to the most useful renditions while maximizing the amount of information we present.  To do this, we need a model that describes what the most relevant information is.

%The zero-one linear program is formulated as follows:
%\begin{equation*}
%\textrm{ argMax }
%\sum_{i=1}^{|R|}
%\sum_{j=1}^{|D|}
%\alpha(r_i, d_j) x_{ij}
%\end{equation*}

%\begin{subnumcases}{
%\textrm{ s.t. } 
%}
%   \forall_{i \leq |R|} \forall_{j \leq |D|} r_{ij} x_{ij} \leq r_{ij} \label{data_rendition_compatability}\\
%   \forall_{i\leq|R|} \sum_{j\leq|D|} x_{ij} = 1 \label{rendition_Cap} \\
%\forall_{j\leq|D|} \sum_{i\leq|R|} x_{ij} \leq 1  \label{unary_data}\\
%\sum_{i\leq|R|} \sum_{j\leq|D|} x_{ij} = %\min(|R|, |D|) \label{complete_fill}
%\end{subnumcases}

%\item[Weighting Information-Rendition Pairings]
Specifically, when deciding how (or whether) to render a piece of information, we must consider three features with the acronym (CIA): the rendition's communicative ability (C), the data's importance (I), and the attention cost (A).
We combine these by subtracting the attention cost ($A$)  from the benefits ($C*I$): 
\begin{equation}
\label{eq::CIA}
\alpha(r_i, d_j) = C(r_i, d_j)*I(d_j)-A(r_i, d_j)
\end{equation}

\begin{description}
\item[Communicative Ability]
Communicative Ability, $C(r,d)$, is the measure of how well a rendition, $r$, communicates a specific type of information, $d$.
%A high level measure of this is already accounted for in the hard constraints of the linear program: if a rendition is incompatible (\ie incapable of communicating a class of information) constraint \ref{data_rendition_compatability} will not be met and the pairing will not be accepted. But our model must support more nuanced expressions of information communication. 
%For instance, there may be many types of renditions that label a coordinate on a map, but each are subject to certain limitations. 
For example, the number of parallel roads that can be represented in a space depends on the width of the lines representing the roads; denser road networks require thinner lines, while sparse networks could make use of more distinguishable, thicker lines. Further, certain attributes of a rendition may be more or less desirable to a specific user, such as the preference for braille or embossed text. 
%In our system, all renditions have a set of attributes, $a \in A_r$, (\eg uses braille, uses raised edges vs uses indented edges). Similarly, users profiles, $U$, and classes of information, $d$, have a set of preferred attributes, $\hat{A}_U,\hat{A}_d$ and weights on those preferred attributes $\beta_a$. 
More formally, the communicative ability of an information-rendition pairing is the weighted sum of the intersection of a rendition's attributes, an information class's preferred attributes, and a user's preferred attributes. 
\begin{equation}
\label{eq::communicability}
C(r,d,U) = \sum_{a\in A_r \cap \hat{A}_d}(\beta_a) +  \sum _{a\in A_r \cap \hat{A}_U}(\beta_a)
\end{equation}
%Attributes can be expressed in many ways, and determining the intersection of a rendition's attributes and preferred attributes is managed through a series of adapter interfaces. For instance, one interface is a Braille-compliant rendition, which requires the rendition to generate related text in braille. A user profile and information maintains a list of relevant adapters (\ie the preferred attributes) mapped to the preferential weight and any parameters of that preference. An example parameter is a preference for lines no thicker than 2 mm or no shorter than 1mm. These parameters are applied to the rendition through the adapter. 

\item[Information Importance]
%Information importance is the simplest measure of an Information-Rendition Pairing and it is central to the adaptive design paradigm our system represents. 
Users know what information is important to them. They know the accessibility features and challenges that affect them most, and what points of interest are most relevant to them. %For this measure, we simply 
Thus, we ask users to rank information. %, the higher the ranking, $I(d_j)$, of a class of information, the higher the weight. If the information is useless or irrelevant to the user the ranking is set to zero which intern makes the weight, $\alpha$, on all rendition pairings with this information zero or less, guaranteeing that the information will not be presented. For efficiency reasons, information classes that are marked as having no importance are excluded from linear program a priori. 
We can pre-populate this with default rankings based on a survey of typical user preferences. 

\item[Attention Cost]
The attention cost of a rendition measures how distracting it is to gather information from it. For example, if a user were using a rendition of a road network to navigate the map in search of a target, say their favorite coffee shop,  the longer it takes to find the target the more information they must keep track of (\textit{e.g.,} where they started, what turns they made, what landmarks they noticed). Tracking all of this information carries an attention cost that would otherwise be spent on the primary search task.

A key problem here is that when the map is being constructed and the attention cost weight is needed, we do not know what the user's target(s) will be, where they will start their search, or what other information is presented that may help or hinder the search task. To address this, we will use a Monte-Carlo simulation of user behavior based on our experimental work.\jm{megan: reference for monte carlo?} %The presentation of that information is, in fact, our goal. Given this level of uncertainty, we use a Monte-Carlo simulation to estimate the average \textit{time} it takes for the user to perform a search given a rendition-information pairing. The probability distribution used to generate this Monte-Carlo simulation is derived from a Markov-chain state model representing the actions a user can take when he or she encounters portions of a rendition. 
%\begin{figure}
%\label{fig::exampleStates}%
%	\begin{tikzpicture}
%        % Add the states
%        \node[state] at (0, 0) (o) {Off Road};
%        \node[state] at (4, 0) (r) {On Road};
%        \node[state] at (2, 4) (i) {Intersection};

%        \draw[every loop]
 %       	(i) edge[loop left] node {} (i)
%        	(i) edge[bend right=20] node %{} (r)
%            (r) edge[bend right=20] node {} (i)
%            (i) edge[bend right=20] node {} (o)
%            (o) edge[bend right=20] node {} (i)
%            (o) edge[loop left] node {} (o)
%            (r) edge[bend right=20] node {} (o)
%            (o) edge[bend right=20] node {} (r)
%            (r) edge[loop right=20] node {} (r);
%    \end{tikzpicture}
%\caption{Example of states of navigating a road network}
%\end{figure}
\end{description}

%Suppose that a user is navigating a rendition of a road network. There are three basic states of the action: (1) moving a finger along the road, (2) encountering an intersection of roads, and (3) moving a finger along the new road. Which particular state the user is in depends on the specific roads paired to the rendition, and their likelihood of moving from one state to another depends on both the particular road network (the information) and the rendition. For example, when the user encounters an intersection, he or she is very likely to continue onto a new road (\eg changing from the Intersection to On-Road state), but which road depends on many factors. Generally, the user is more likely to move in the same direction and increasingly less likely to turn all the way around. If the rendition presents very thin roads, the user is more likely to travel in a direction that they don't feel the roads any more, moving into the "Off-Road" state. We model the probability of this state change by selecting a random angle between $-\pi$ and $\pi$ from a normal distribution with a mean direction of 0 (no change in direction from the approaching road). If the user travels in that direction but could still feel a road (based on the width of the finger and thickness of the rendition), then the angle will be modified to continue along that road, otherwise they will move randomly in that general direction off-road until they find a new road to follow.  

%Each state change in the Markov-chain takes about step of one finger width, and we use a state change as a unit of time. For each simulation in the Monte-Carlo model, the user takes a "walk" across the map starting at a random point and moving towards a target. The starting points and targets are selected from a probability distribution where areas denser with information are more likely to be selected. The Markov-chain determines the walk that they take. We count the number of steps in the walk it takes to find the target given the rendition and information. The average number of steps over many models is our attention cost, meaning the more difficult (the longer it takes) it is to find information using a rendition-information pairing, the greater the attention cost and the poorer the pairing. 

%In terms of implementation, each rendition, $R$, has a set of states $S_R$. The probability of entering a state, $s_i\in S_R$, is dependent on the data, $D$, paired to $R$ and the state it is entering from, $s_{i-1}$: $P(s_i | D, s_{i-1})$, A walk over the map, $W$, starts from a starting point/state $s_1$ and we randomly change the state based on the probability distribution of all possible next-states. With each state we move the finger, $f$, a one finger width, $w_f$ in the direction dictated by the current state. The attention cost of a walk, $A(W)$, is the number of states need to get the point $f$ within $w_f$ of the target point $t$.The attention cost of a pairing, $A(R,D)$ is equal to the average attention cost of all of $N$ simulated walks.

%\begin{equation}
%W \subset S_R | s_1...s_{|W|}
%\end{equation}
%\begin{equation}
%A(W) = |W|
%\end{equation}
%\begin{equation}
%A(R,D) = \frac{\sum_{i=1...N} A(W_i)}{N}
%\end{equation}

\subsubsection{Proposed Development}

An optimization algorithm can use this information to decide on the optimal map. In optimization terms, $C*I-A$ is an \textit{objective function}, which can calculate a score for a possible map. Off-the-shelf algorithms can solve for the best solution (map in our case). For this problem, we can then  formulate the optimization as a zero-one linear program interpretation based on the Generalized Assignment Problem 
\cite{kuhn1955hungarian}. 

During this Development Stage we will also develop proper system tests to assess algorithmic accuracy/validity, costing for  enhanced Pedestrian Accessibility, testing that feasible routes are produced. Tests would be performed on  an array of Test addresses and resulting locations, on an array of user-defined choices and preferences, and test on 
intended and randomly selected street segments in Seattle (to test for the presence and accuracy of sidewalk information)



\subsubsection{Validation}
We will take a user-centered, iterative approach to the design of our mapping solution. 
All of human-in-te-loop studies will be conducted with the population we are serving, Deaf-blind users. 
PI Caspi has extensive experience working with this population \jm{Anat: true?} and will  
help to ensure that we have access to a sufficient number of participants.  


\subsubsection{Map Study 2: Technical Validation of Speed and Robustness}
\label{sec:technical-validation}
Our first study will focus on validating the algorithm's {\em speed} and {\em robustness}. This is solely a technical validation. The algorithm we are using is well understood, however in any optimization, the goodness of the result and speed are both influenced by the goodness of the objective function. Thus it is important to empirically validate the quality of the results.

\paragraph{Sample} 
Our sample will be a random selection of 1000 regions taken from the Seattle metro region. Our goal is to demonstrate successful coverage of a wide range of possible regional features. Since the objective function may differ from user to user, we will run the algorithm through a range of optimization parameters for each of these regions. 


\paragraph{Environment}
We will run the optimization on all samples on a server machine similar to the environment in which our back-end product will ultimately run. During runs, we will collect information about the length of time it takes to reach a solution, and errors that occur during the optimization process, and features of the final result of the optimization, including its score on the objective function. 

\paragraph{Test Trials}
As stated above, we will collect data on the performance of the optimization algorithm across a large set of regions and objective function parameters. We will validate that the optimization algorithm can run in real time or requires minimal wait time (our {\em speed} goal), and that it  produces a valid, 3D printable result in each case ({\em robustness} goal)

\subsubsection{Map Study 2: Heuristic Validation of Accuracy}
\label{sec:technical-validation}
Our second  study will focus on validating the algorithm's {\em accuracy}. Specifically, we will
show that the tangible maps produced by the optimization are correctly optimizing for the features specified in the 
objective function and that they adequately balance legibility and feature richness for a variety of preferences.  

\paragraph{Sample} 
For this study, we will create a sample of 20 hand-curated maps using the process described in~\ref{sec:prev-devel-access}. These hand-curated maps represent optimal solutions as designed by people, and we will use them as comparison points for the maps that our algorithm generates.  We will then create a set of 100 3D printed maps using 5 separate objective functions representing a range of user needs.

Our study will also depend on an expert sample of orientation and mobility experts, who can help to assess the value of the maps we created before we test them with end users. 

\paragraph{Environment}
As wit the previous study, we will run the optimization on all samples on a server machine similar to the environment in which our back-end product will ultimately run. During runs, we will collect information about the length of time it takes to reach a solution, and errors that occur during the optimization process, and features of the final result of the optimization, including its score on the objective function. In addition, we will print out the result of the optimization on an Ultimaker desktop 3D printer.

\paragraph{Test Trials}
We will inspect the 3D printed maps from the hand curated and optimized samples and assess how well they each balance for legibility and feature richness. This will be a heuristic evaluation by experts (orientation and mobility experts) focused on assessing how well the final maps meet specific goals set in the objective function. To accomplish this, we will present the experts with English versions of the five objective functions we are testing such as 'favors streets with sidewalks and lower environmental stress (e.g., lower speed limits and traffic volume).' 

\subsubsection{Map Study 3: Map Understanding by End Users With and Without Optimization}
\label{sec:lab-tests}
Our third study will focus on demonstrating the benefits to end-user understanding of optimization over standard map designs in a controlled fashion. For this study, we will use pre-defined maps that focus on neighborhoods familiar to participants. We will ask participants to indicate the location of a well known landmark, and to describe the region around the landmark. 

\paragraph{Sample}
Our primary goal in this study is to understand whether someone without vision can correctly use the tactile information in the maps. Although this project aims to improve the transit experience for Deaf-blind participants, the specific issue of map interpretability can be tested just as effectively with people who are Blind, which is our intention. In this way, we will reduce the burden on the smaller local population of Deaf-blind individuals by focusing on them in later studies, where their unique perspectives are more critical.

\paragraph{Environment}
The study will be conducted in a lab environment. Because we are focusing on readability, this study will be more effective if we can control for events that might arise in the field, such as distractions.

\paragraph{Test trials}
We will ask participants to describe features of a route between two landmarks. We will then have them perform the same tasks using an optimized map. \jm{how do we ensure familiarity? do we care about route finding? How do we make sure the optimization is relevant to them? How do we train them on all the renditions' meanings?+}



\subsubsection{Map Study 4: Navigation Benefits of Optimization}
\label{sec:field-map}
%\jm{I think we also want a study of them choosing a route of their choice and using our system?}
The goal of our final study is to demonstrate the usefulness of our maps in the field. For this study, we want to increase realism so we will be more likely to discover unexpected variables that may impact the success of the product. 

\paragraph{Sample}
We will work with Deaf-blind individuals to test the maps. 


\paragraph{Environment}
The study will take place in an outdoor environment where all of the complexities of real world navigation come into play. In addition, we will study map use for each participant in a location of their choosing.

\paragraph{Test Trials}
For the first trial, participants will be asked to identify two known locations within walking distance of each other. We will create a map that includes both, optimized based on participant preferences. Participants will navigate from one location to the other, using the map. This represents a relatively easy task, since it is a known navigation task, but will give participants a chance to tell us something about how the map enhances their experience.

For the second trial, participants will be asked to explore the region around a new location. \jm{Anat: does this make sense? What is the normal process by which participants orient themselves?}.

