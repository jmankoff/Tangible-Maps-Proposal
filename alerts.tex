
\subsubsection{Background and State of the Art}
\ac{fix all the references here}

Studies have shown, for both familiar and unfamiliar environments providing information about an area ahead of time can enhance independence in travelers with low vision and blindness (Quinones et al. 2011, Campbell et al. 2014).  More information has been shown to empower public transit riders to attempt unfamiliar trips (Campbell et al. 2014), but inaccurate and incomplete data has been a significant source of frustration in facilitating this kind of independent travel (Bonnar C et al. 2015).  Pedestrians that require an assistive mobility device prior to a trip often use digital tools such as Google Street View, in order to get a more complete understanding of the environments that they are planning to traverse and to hopefully identify any unsurpassable obstacles ahead of time (Hara, 2016).   However, digital mapping tools are generally not accessible to a broad spectrum of users that would benefit from more a granular understanding of space, prior to entering. Individuals that are vision-impaired require the assistance of a person with sight to utilize Google Street View as a method for gaining insight into an area before experiencing it.  Additionally, current tools generally lack the ability to reflect recent changes in the environment, whether those be human induced (construction) or otherwise (weather) (Quinones et al. 2011).


\subsubsection{Previous Development}

\subsubsection{Proposed Development}

\subsubsection{Validation}
Success will provide unprecedented levels of access to environment exploration and to real-time automated transit information for people who have both limited vision and hearing. Specifically, successful tactile maps would allow users to handle both emergent and serendipitous changes in overall path and destination, and with practice may assist in just-in-time navigation and wayfinding.
The personalized event notifications, individualized trip alerts could assist users in avoiding certain pedestrian areas, in better understanding real-time disaster, and failure recovery.