\ac{this section should completely motivate the optimization problem and doesn't currently.
The important points are:
maps provide information that closes the informational gap for people about the environment
**current GPS solutions do not offer the information in Deaf-blind consumable format
** current embossed maps are useful, but are costly to produce and difficult to get a hold of. Also, they are limited in depth dimensionality (the embossing is either YES/NO) so it has limited expressivity as a medium.
** tactile maps can in principle contain continuous depth and therefore increase the expressivity and informational content in the maps
** tactile maps that have been produced with accessibility in mind lack in two major ways: (1) they do not take a pedestrian-centric approach to the data presented (they just take road automaps and convert them to tactile tiles). There's little value for a Deaf-blind person to have a raised edge for the automobile road. As we've learned from our pilot, DB population want to know very specific attributes of the SIDEWALKS and the SURFACES that they might meet on the way, also specific attributes of the pedestrian crossings are significantly missing). The one good thing these maps do is show the outline of the buildings. However! Since the tactile maps make the building footprints RAISED, they significantly hamper the users' ability to inspect the remaining environment in the 'negative spaces'. Much like in some works of art, the actually relevant portion of the touch mapper maps are in the negative spaces, and are left for interpretation by the user. We need to change this in order to provide useful tactile maps, enriched with relevant, pedestrian-centric information.
The problem we need to tackle is the viable combining of 
3D printed objects (with actual physical manifestation and the constraints of users being able to read them)
and 
relevant pedestrian-centric information (with tactile features that appropriately highlight what is significant to a DB person).

Usability is one of our main concerns and maps of this nature can get dense in high density urban areas. Therefore, we must be careful that the paths and what's printed on our tiles is optimized for the most salient information content. This is unfortunately, not a simple hierarchical decision tree, but requires careful algorithmic considerations.

}



