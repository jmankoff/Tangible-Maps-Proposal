The greatest challenge in mobility and orientation for visually impaired people is to understand the spatial layout of the targeted environment. Maps are an important tool for sighted navigation, but inherently visual. One solution that has a long history in the Blind and Blind-deaf community is tactile maps. Common approaches include embossing, printing ind Braille, and "swell" (microcapsule) paper. These approaches can be used to create a two-layer tactile visualization from an image (\textit{e.g.}, \cite{miele2006talking}). However such machines can be very costly \cite{rice2005design}, are not readily available, and lack the expressive range of a solution that can vary more in height. While vacuum-forming has more height, it depends on a pre-existing form, which limits its customizability. Finally, the design of a tactile map requires expertise to be easily understandable \cite{tatham1991design}.

With the advent of consumer-grade fabrication technologies, 3D printed and other custom fabricated maps have been receiving increasing attention. For example, it is now possible to requisition a custom, laser-cut topographical map on ETSY \cite{etsy} or purchase custom 3D printed topographic models through companies such as Sightline \url{sightlinemaps.com} or PrintMyRoute \url{www.printmyroute.xyz}. However, these services do not include features intended to support Blind or Deaf-blind users, and do not support activities such as route finding and landmark identification as a result. 

The accessibility community has added its own spin to these technologies. For example TouchMapper \url{touch-mapper.org} is a free service that will create a tactile map using embossing or 3D printing and help connect you to existing online printing services or print it yourself.  This service focues on showing streets, and can be customized for size, inclusion of buildings and other features. However it too does not include route finding or landmark identification. 

From a research perspective, one open problem is combining tactile maps and smartphones. By embedding capacitive touch sensing capabilities in tactile maps, it is possible to provide audio feedback about the region someone is touching \cite{taylor2016customizable, rusu2010semantic,gotzelmann2016lucentmaps}. 

Another open question is which cartographic features should be highlighted in tactile maps \cite{haberling2008proposed}. While tactile interaction has been studied extensively (\textit{e.g.}, see this review \cite{o2015designing}), the specific design considerations for tactile maps are not as well understood, particularly for Deaf-blind individuals. In addition, there is no computational model that takes these variables into account exists.

Finally, questions remain about the ability to of tactile maps to support route finding (as opposed to orientation). For example, Gual \textit{et al.} found that standard 3D printed maps can improve memorization in route finding, but could not be used autonomously without collaborator support \cite{gual2012visual}. However, they did not explore a wide range of tactile variables to support interpretation. 

To summarize, tactile maps are not new to the cartographic record \cite{xx}. Their value in facilitating orientation and navigation for the low vision and blind communities has been well established \cite{XXX}. However, their scope and availability has been greatly limited in the past by high production costs and limited interest from fields traditionally invested in map making and design.  3D printing has significantly reduced the costs of producing such maps, but to our knowledge no existing product has enhanced 3D printed maps with optimization and customization, making our product and important addition to this space. 

