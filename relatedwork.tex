\jm{brandon's text. Write a similar intro paragraph: Tactile maps have a long history that predates 3D printing as a
commercially available technology. Embossers and microcapsule
paper can be used to automatically create maps from computergenerated images [23]. However, these techniques provide only
two layers of depth and require expensive equipment. Braille
embossers start at \$1800 and can range up to \$80,000 for highspeed machines [2]. Microcapsule, or swell paper, printers can be
had for \$1350 and also require specialized paper that costs more
than a \$1 per sheet [3]. Another production method, vacuum
forming, offers a wider range of possible tactile features.}

\ac{
Also not my text:

Mobility and orientation are  among the greatest challenges for visually impaired people.  One reason that explains these issues is that visually impaired users in general usually exchange or are given verbal descriptions of itineraries, which may help them to find their way, but do not provide them with  any  clue  about the  spatial  layout  of  the targeted  environment. GPS-based  systems,  although  facilitating  navigation,  raise  the  same  issue.  Sighted  people  usually acquire information about a spatial environment through visual perception or by using geographical maps.  However,  maps  are  essentially  visual  and  thus  inherently  inaccessible  for visually  impaired people.  And  weak access to  maps has drastic  consequences  on mobility  and  education, but  more generally on personal and professional life, and can lead to social exclusion (Passini & Proulx, 1988).  Beyond  orientation  and  mobility  purposes,  maps  are  very  useful  tools  to  explore  and  analyze geographical  data  and  to  acquire  general  knowledge  about  many  subjects  such  as  demography, geopolitics,  history,  etc. They  are  also often  used in  the classroom  for this  purpose. As  stated  by O’Modhrain  et  al.  (2015),  “there  is  an  immediate  need  for  research  and  development  of  new technologies to provide non-visual access to graphical material. While the importance of this access is  obvious  in  many  educational,  vocational,  and  social  contexts  for  visually  impaired  people,  the diversity of  the user group,  range  of available technologies, and  breadth  of tasks to  be  supported complicate the research and development process”.  In specialized educational centers for visually  impaired people,  tactile  maps are  commonly used  to give visually impaired students access to geographical representations. However, these materials are rarely used or available outside  of  school, because their  production  is a  costly and time-consuming process (Rice, Jacobson, Golledge, & Jones, 2005). To create a tactile map, one of the most common methods is to print it on a special paper, called “swell paper” (synonyms are “microcapsule paper” or “heat sensitive  paper”), which contains microcapsules  of alcohol  in  its coating.  When the  paper  is heated, the microcapsules expand and create relief over black lines (Figure 1.a). The resulting maps, called raised-line maps, can be perceived by touch. But they are also visual maps, making it possible to  share  information  between  blind  and  (partially)  sighted  people.  Raised-line  maps  are  usually prepared with a computer, which makes it possible to print and fuse several copies of the same map.  Another techniques, called vacuum-forming or thermoforming, consists in placing a sheet of plastic over a  master made of a variety of textured materials.  When it is heated in a vacuum, the sheet is permanently  deformed  according  to  the  master.  Hand-crafted  techniques  can  also  be  used  to produce maps and  other  graphics.  For example,  for orientation and mobility lessons,  teachers  and students construct itineraries or local maps, by progressively placing magnets over a magnetic board (Figure  1.b).  Students are  sometimes asked  to replicate  the  construction, so  that the  teacher can check  that  the  itinerary  has  been  memorized.  Small-scale  models  made  out  of  wood  also  exist, alongside  graphics  made  out  of  paper,  cardboard,  ropes,  etc.  (see  Figure  1.c).  Edman  (1992) presented a comprehensive summary of production techniques for accessible maps. 
   
   
   Although  tactile  maps  are  efficient  for  acquisition  of  spatial  knowledge,  they  present  several limitations  and  issues.  As  stated  by  Rice  et  al.  (2005)  the  production  of  tactile  maps  is  time consuming and expensive. In addition, tactile maps must be produced by a tactile graphics specialist who knows how to present information so that it can be perceived by touch (Tatham, 1991). Other critiques  concern  the  number  of  elements  that  can  be  displayed  and  the  accuracy  of  the  map content.  Indeed,  because  of  the  perceptual  limits  of  the  tactile  modality,  less  details  can  be represented on a tactile map  than  on  a visual map. Furthermore, once  a tactile  map is printed,  its content is static and cannot be adapted dynamically. Tactile maps are then quickly getting outdated (Yatani,  Banovic, &  Truong,  2012).  In  addition,  the  use  of  braille  labels  is  an  issue.  Only  a  small percentage  of  visually  impaired  people  read  braille  (National  Federation of  the Blind,  2009);  and braille is not so convenient when compared to printed text. Text on visual maps can be written with different font sizes and styles. It can be rotated to fit in open spaces. Upper and lower cases, as well as color, can be used to highlight important  items.  In  contrast,  braille  lacks all of these possibilities and needs a lot of  space  because it is  fixed  in  size, inter-cell spacing and orientation (A. F. Tatham, 1991). Due to the lack of space, braille abbreviations are commonly used on tactile maps, which are then explained in a legend accompanying the map. Because the reader must frequently move hands between  the  map and  the  legend, it  disrupts the  reading process  (Hinton,  1993). Finally,  because they  are  not  interactive,  tactile  maps  cannot  provide  advanced  functionalities  such  as  panning, zooming, annotating, computing itineraries or distances, and using filters to facilitate the exploration.
}

The greatest challenge in mobility and orientation for visually impaired people is to understand the spatial layout of the targeted environment. 
Although  tactile  maps  have proven comparatively efficient  for  acquisition  of  spatial  knowledge,  they  present  several limitations  and  issues.

With the advent of consumer-grade fabrication technologies, 3D printed and other custom fabricated maps have been receiving increasing attention. For example, it is now possible to requisition a custom, laser-cut topographical map on ETSY \cite{etsy} or purchase custom 3D printed topographic models through companies such as Sightline \url{sightlinemaps.com} or PrintMyRoute \url{www.printmyroute.xyz}. However, these services do not include features intended to support Blind or Deaf-blind users, and do not support activities such as route finding and landmark identification as a result. 

The accessibility community has added its own spin to these technologies. For example TouchMapper \url{touch-mapper.org} is a free service that will create a tactile map using embossing or 3D printing and help connect you to existing online printing services or print it yourself.  This service focues on showing streets, and can be customized for size, inclusion of buildings and other features. However it too does not include route finding or landmark identification. 

From a research perspective, one open problem is combining tactile maps and smartphones. By embedding capacitive touch sensing capabilities in tactile maps, it is possible to provide audio feedback about the region someone is touching \cite{taylor2016customizable, rusu2010semantic,gotzelmann2016lucentmaps}. 

Another open question is which cartographic features should be highlighted in tactile maps \cite{haberling2008proposed}. Although some work has explored this design space, it did not focus on the needs of Deaf-blind individuals, and no computational model that takes these variables into account exists.

Finally, questions remain about the ability to of tactile maps to support route finding (as opposed to orientation). For example, Gual \textit{et al.} found that standard 3D printed maps can improve memorization in route finding, but could not be used autonomously without collaborator support \cite{gual2012visual}. However, they did not explore a wide range of tactile variables to support interpretation. 

To summarize, tactile maps are not new to the cartographic record \cite{xx}. Their value in facilitating orientation and navigation for the low vision and blind communities has been well established \cite{XXX}. However, their scope and availability has been greatly limited in the past by high production costs and limited interest from fields traditionally invested in map making and design.  3D printing has significantly reduced the costs of producing such maps, but to our knowledge no existing product has enhanced 3D printed maps with optimization and customization, making our product and important addition to this space. 

